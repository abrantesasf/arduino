%%%%%%%%%%%%%%%%%%%%%%%%%%%%%%%%%%%%%%%%%%%%%%%%%%%%%%%%%%%%%%%%%%%%%%%%%%%%%%%%
\section{Materiais}
\label{sec:materiais}

Nosso Arduino Standalone terá, na protoboard, os seguintes grandes componentes:

\begin{enumerate}
  \item \textbf{Alimentação}: usaremos uma bateria ou fonte de alimentação de
        \qty{9}{\volt} mas o ATmega328P suporta, no máximo
        \qty{5.5}{\volt}. Precisaremos então de um circuito regulador de tensão
        que receba os \qty{9}{\volt} e entregue \qty{5}{\volt} para o
        microcontrolador.
  \item \textbf{Microcontrolador}: é o ATmega328P em si. Você pode remover o
        microprocessador do Arduino, ou comprar um novo.
  \item \textbf{Clock externo}: usaremos um circuito de clock externo de
        \qty{16}{\mega\hertz} com um cristal oscilador.
  \item \textbf{Circuito de reset}: um botão na protoboard permitirá que o
        microcontrolador seja reiniciado rapidamente, sem a necessidade de
        desconectar a fonte de energia.
\end{enumerate}

A organização desses grandes componentes na protoboard será aproximadamente a
exibida na Figura~\ref{fig:orgnizacao_componentes}, a seguir:

\begin{figure}[H]
\centering
\caption{Organização geral dos componentes}
\label{fig:orgnizacao_componentes}
\vspace{-0.2cm}
  \includegraphics[scale=0.18]{imagens/protoboard3.jpg}
%\footnotesize{Fonte: xxx}
\end{figure}

Os materiais e componentes necessários são os seguintes:

\begin{itemize}
  \item \textbf{Alimentação}: a lista a seguir e a
        Figura~\ref{fig:comp-alimentacao} mostram os componentes que
        utilizaremos para a alimentação do microcontrolador. Alguns itens são
        opcionais mas recomendados neste estágio inicial de seu aprendizado.
        \begin{itemize}[noitemsep]
          \item 1 regulador linear de tensão L7805CV
                % www.newark.com/stmicroelectronics/l7805cv/ldo-fixed-5v-1-5a-0-to-125deg/dp/26M0575
          \item 1 diodo 1N4007
                % www.newark.com/multicomp-pro/1n4007/diode-standard-recovery-1a-1kv/dp/65W8781
          \item 2 capacitores eletrolíticos de \qty{47}{\micro\farad}
                % newark.com/chemi-con/esmg500ell470mf11d/aluminum-electrolytic-capacitor/dp/23K5056
          \item 2 capacitores cerâmicos de \qty{100}{\nano\farad}
                % ???
          \item 1 LED vermelho de \qty{5}{\mm} (opcional, apenas para indicar
                que a protoboard está energizada)
                % www.newark.com/cree-led/c503b-rcn-cx0y0aa1/led-round-5mm-red-5-1cd-624nm/dp/08R2995
          \item 1 resistor de \qty{680}{\ohm} (opcional, apenas para indicar que
                a protoboard está energizada)
                % www.newark.com/multicomp-pro/mf50-680r/metal-film-resistor-680-ohm-500mw/dp/38K5245
          \item 1 conector borne de duas vias
                % www.newark.com/lumberg/kre-03/wire-to-board-terminal-block-3/dp/25M9958
          \item 1 clip para bateria de \qty{9}{\volt} e 1 bateria de \qty{9}{\volt}
                % www.newark.com/keystone/233/battery-strap-9v-wire-lead/dp/22C4351
                % www.newark.com/energizer/522bp-2/alkaline-zn-mno2-battery-9v/dp/17C5580
          \begin{figure}[H]
          \centering
          \caption{Componentes para a alimentação}
          \label{fig:comp-alimentacao}
          \vspace{-0.2cm}
            \includegraphics[scale=0.9]{imagens/alimentacao.jpeg}
          %\footnotesize{Fonte: xxx}
          \end{figure}
        \end{itemize}
        Se você não quiser usar uma bateria, pode utilizar uma fonte de
        alimentação de \qty{9}{\volt} (\qty{1}{\ampere})\footnote{Estou usando
        uma fonte que fornece corrente de até \qty{1}{\ampere}, mas você pode
        utilizar qualquer outra fonte de \qty{9}{\volt} que tiver disponível.},
        com um adaptador com borne de duas vias conforme a
        Figura~\ref{fig:fonte9v} (na página~\pageref{fig:fonte9v}), para
        conectar a fonte à protoboard.
        % www.newark.com/pro-elec/28-19355/9vdc-1a-regulated-ac-power-adapter/dp/62X7077
        Para mantermos nosso Arduino Standalone o mais portátil possível
        usaremos uma bateria comum mesmo.
        \begin{figure}[H]
        \centering
        \caption{Alternativa para alimentação}
        \label{fig:fonte9v}
        \vspace{-0.2cm}
        \includegraphics[scale=0.8]{imagens/alimentacao2.jpeg}
        %\footnotesize{Fonte: xxx}
        \end{figure}
  \item \textbf{Microcontrolador} e \textbf{circuito de reset}: precisaremos dos
        materiais ilustrados na lista abaixo (e na Figura~\ref{fig:atmega328p}):
        \begin{itemize}
          \item 1 microcontrolador ATmega328P-PU (você pode retirar,
                com muito cuidado, o microcontrolador de sua placa Arduino
                atual, ou comprar um novo microcontrolador)
          \item 1 indutor axial de \qty{10}{\micro\henry}
          \item 1 resistor de \qty{10}{\kilo\ohm}
          \item 1 diodo 1N4148
          \item 3 capacitores cerâmicos de \qty{100}{\nano\farad}
          \item 1 push button
        \begin{figure}[H]
        \centering
        \caption{ATmega328P-PU e circuito de reset}
        \label{fig:atmega328p}
        \vspace{-0.2cm}
        \includegraphics[scale=1.0]{imagens/microcontrolador.jpeg}
        %\footnotesize{Fonte: xxx}
        \end{figure}
        \end{itemize}
  \item \textbf{Clock externo}: para montar o circuito de clock precisamos dos
        materiais a seguir (ilustrados na Figura~\ref{fig:clock}):
        \begin{itemize}[noitemsep]
          \item 1 cristal oscilador de \qty{16}{\mega\hertz}
          \item 2 capacitores cerâmicos de \qty{22}{\pico\farad}
          \item 1 resistor de \qty{1}{\mega\ohm}
        \end{itemize}
        \begin{figure}[H]
        \centering
        \caption{Clock externo}
        \label{fig:clock}
        \vspace{-0.2cm}
        \includegraphics[scale=1.0]{imagens/clock.jpeg}
        %\footnotesize{Fonte: xxx}
        \end{figure}
\end{itemize}

Além dos materiais e componentes principais listados anteriormente, também será
necessário uma protoboard de 830 furos (eu prefiro os modelos que têm quatro
linhas de energia separadas), ferramentas diversas (alicates de corte,
decapadores de fio, alicates de bico fino, pinças) e, se disponível, um
multímetro para verificar tensão, corrente e resistência (Figura~\ref{fig:}):

\begin{figure}[H]
\centering
\caption{Ferramentas úteis}
\label{fig:}
\vspace{-0.2cm}
  \includegraphics[scale=1.0]{imagens/ferramentas.jpeg}
\end{figure}

Por fim precisamos de fios para as conexões na protoboard e aqui há uma regra
clara: \textbf{não use} fios jumper flexíveis, como aqueles que são comumente
utilizados em kits de Arduino para iniciantes. Fios jumper flexíveis são
grandes, ficam sobrando e fazem uma tremenda confusão da protoboard. Para montar
um Arduino Standalone de modo ``profissional'' precisamos usar \textbf{fios
rígidos sólidos} para eletrônica, com bitola entre
\qty{24}{AWG}\footnote{\ingles{American Wire Gauge} (AWG): é uma escala
logarítmica americana para a identificação das bitolas de fios rígidos,
especialmente os fios utilizados em eletrônica.} e \qty{22}{AWG} (entre
\qty{0.511}{\mm} e \qty{0.644}{\mm}). Tenha sempre à mão um estoque de
diferentes cores desses fios rígidos, conforme a Figura~\ref{fig:fios-rigidos}
abaixo:

\begin{figure}[H]
\centering
\caption{Fios rígidos 22 AWG}
\label{fig:fios-rigidos}
\vspace{-0.2cm}
  \includegraphics[scale=0.8]{imagens/cabos_rigidos3.jpeg}
%\footnotesize{Fonte: xxx}
\end{figure}

