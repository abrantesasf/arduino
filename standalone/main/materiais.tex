%%%%%%%%%%%%%%%%%%%%%%%%%%%%%%%%%%%%%%%%%%%%%%%%%%%%%%%%%%%%%%%%%%%%%%%%%%%%%%%%
\section{Materiais}
\label{sec:materiais}

Nosso Arduino Standalone terá, na protoboard, os seguintes grandes componentes:

\begin{enumerate}
  \item \textbf{Alimentação}: usaremos uma bateria ou fonte de alimentação de
        \qty{9}{\volt} mas o ATmega328P suporta, no máximo
        \qty{5.5}{\volt}. Precisaremos então de um circuito regulador de tensão
        que receba os \qty{9}{\volt} e entregue \qty{5}{\volt} para o
        microcontrolador.
  \item \textbf{Microcontrolador}: é o ATmega328P em si. Você pode remover o
        microprocessador do Arduino, ou comprar um novo.
  \item \textbf{Clock externo}: usaremos um circuito de clock externo de
        \qty{16}{\mega\hertz} com um cristal oscilador.
  \item \textbf{Circuito de reset}: um botão na protoboard permitirá que o
        microcontrolador seja reiniciado rapidamente, sem a necessidade de
        desconectar a fonte de energia.
\end{enumerate}

A organização desses grandes componentes na protoboard será aproximadamente a
exibida na figura~\ref{fig:orgnizacao_componentes}, a seguir:

\begin{figure}[H]
\centering
\caption{Organização geral dos componentes}
\label{fig:orgnizacao_componentes}
\vspace{-0.2cm}
%\fbox{
  \includegraphics[scale=0.18]{imagens/protoboard3.jpg}
%}\
%\footnotesize{Fonte: xxx}
\end{figure}

Os materiais e componentes necessários são os seguintes:

\begin{itemize}
  \item \textbf{Alimentação}: a lista a seguir e a
        figura~\ref{fig:comp-alimentacao} mostram os componentes que
        utilizaremos para a alimentação do microcontrolador. Alguns itens são
        opcionais mas recomendados neste estágio inicial de seu aprendizado.
        \begin{itemize}[noitemsep]
          \item 1 regulador linear de tensão L7805CV
                % www.newark.com/stmicroelectronics/l7805cv/ldo-fixed-5v-1-5a-0-to-125deg/dp/26M0575
          \item 1 diodo 1N4007
                % www.newark.com/multicomp-pro/1n4007/diode-standard-recovery-1a-1kv/dp/65W8781
          \item 2 capacitores eletrolíticos de \qty{47}{\micro\farad}
                % newark.com/chemi-con/esmg500ell470mf11d/aluminum-electrolytic-capacitor/dp/23K5056
          \item 2 capacitores cerâmicos de \qty{100}{\nano\farad}
                % ???
          \item 1 LED vermelho de \qty{5}{\mm} (opcional, apenas para indicar
                que a protoboard está energizada)
                % www.newark.com/cree-led/c503b-rcn-cx0y0aa1/led-round-5mm-red-5-1cd-624nm/dp/08R2995
          \item 1 resistor de \qty{680}{\ohm} (opcional, apenas para indicar que
                a protoboard está energizada)
                % www.newark.com/multicomp-pro/mf50-680r/metal-film-resistor-680-ohm-500mw/dp/38K5245
          \item 1 conector borne de duas vias
                % www.newark.com/lumberg/kre-03/wire-to-board-terminal-block-3/dp/25M9958
          \item 1 clip para bateria de \qty{9}{\volt} e 1 bateria de \qty{9}{\volt}
                % www.newark.com/keystone/233/battery-strap-9v-wire-lead/dp/22C4351
                % www.newark.com/energizer/522bp-2/alkaline-zn-mno2-battery-9v/dp/17C5580
          \item 1 fonte de alimentação de \qty{9}{\volt} (\qty{1}{\ampere})
                (opcional: se você não quiser utilizar a bateria pode utilizar
                uma fonte como alimentação)
                % www.newark.com/pro-elec/28-19355/9vdc-1a-regulated-ac-power-adapter/dp/62X7077
          \begin{figure}[H]
          \centering
          \caption{Componentes para a alimentação}
          \label{fig:comp-alimentacao}
          \vspace{-0.2cm}
            \includegraphics[scale=0.7]{imagens/alimentacao.jpeg}
          %\footnotesize{Fonte: xxx}
          \end{figure}
        \end{itemize}
\end{itemize}
