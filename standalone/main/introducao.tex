%%%%%%%%%%%%%%%%%%%%%%%%%%%%%%%%%%%%%%%%%%%%%%%%%%%%%%%%%%%%%%%%%%%%%%%%%%%%%%%%
\section{Introdução}
\label{sec:intro}

Depois de algum tempo utilizando o Arduino e ganhando experiência em diversos
projetos interessantes com sensores e atuadores, você começa a sentir
necessidade de entender realmente como essa placa funciona e como o
microcontrolador realiza a mágica de transformar o código de seu programa em
ações no mundo real.

Montar um ``Arduino Standalone'' (usar apenas o microcontrolador --- geralmente
o ATmega328P --- na protoboard sem a placa do Arduino) é um verdadeiro
\textbf{rito de passagem} no aprendizado de eletrônica, programação e sistemas
embarcados: é o momento em que você deixa de ser apenas um usuário do Arduino
para entender como a engenharia por trás dele funciona.

Pense no Arduino como uma bicicleta com rodinhas de treinamento: você consegue
fazer muitas coisas legais e interessantes mas chega um momento em que, para
avançar, as rodinhas de treinamento precisam ser removidas. Montar seu primeiro
Arduino Standalone equivale à seu primeiro passeio de bicicleta sem as rodinhas
de treinamento.

Existem diversas razões pelas quais isso é interessante e importante:

\begin{itemize}
  \item \textbf{Desmistificação do hardware}: quando usamos o Arduino, muitas
        coisas estão pré-configuradas e ocultas para nós. Ao montar o circuito
        independente na protoboard você aprenderá na prática a função de
        diversos componentes essenciais como o cristal oscilador, a regulação de
        tensão para a alimentação do microcontrolador e a montagem do circuito
        de reset.
  \item \textbf{Eficiência energética}: a placa do Arduino possui componentes
        que constantemente consomem energia (LEDs, conversor USB-Serial) mesmo
        que seu código não esteja fazendo nada. Ao eliminar todos os componentes
        não essenciais e configurar corretamente o microcontrolador (usando o
        modo de \ingles{sleep}) você pode fazer seu Arduino Standalone funcionar
        por vários meses usando apenas duas pilhas AA (um Arduino completo
        acabaria com as pilhas em alguns dias).
  \item \textbf{Custo e permanência}: imagine que você criou um pequeno sistema
        de automação para o portão da garagem de sua casa. Você deixaria seu
        Arduino preso lá no portão para sempre? Correndo o risco de um fio se
        desconectar com o tempo? Você vai perder uma placa relativamente cara e
        que tem diversas outras possibilidades de usos além do portão?
        Provavelmente não. O melhor é que você utilize apenas o microcontrolador
        e os componentes básicos para fazer seu sistema de automação funcionar
        e, ao montar o Arduino Standalone, você aprenderá a como fazer isso.
  \item \textbf{Transição para o produto final}: nenhum produto final acabado
        tem um Arduino colado dentro dele. Produtos reais utilizam apenas o
        microcontrolador e os demais componentes eletrônicos soldados em uma
        placa de circuito impressa (PCB\footnote{Do inglês: ``\ingles{Printed
        Circuit Board}''.}) fabricada profissionalmente. Aprender a montar um
        Arduino Standalone é um passo intermediário obrigatório para que, no
        futuro, você aprenda a projetar suas próprias PCB profissionais.
  \item \textbf{Flexibilidade}: ao usar o microcontrolador independente você não
        fica preso aos \qty{5.0}{\volt} e \qty{16}{\mega\hertz} do Arduino:
        você pode rodar o chip em \qty{3.3}{\volt} e se comunicar com sensores
        modernos sem conversores de nível, ou pode usar o clock interno de
        \qty{8}{\mega\hertz} do microcontrolador e liberar mais dois pinos
        digitais extras para uso.
\end{itemize}
