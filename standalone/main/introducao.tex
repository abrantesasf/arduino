%%%%%%%%%%%%%%%%%%%%%%%%%%%%%%%%%%%%%%%%%%%%%%%%%%%%%%%%%%%%%%%%%%%%%%%%%%%%%%%%
\section{Introdução}
\label{sec:intro}

Depois de algum tempo utilizando o Arduino e ganhando experiência em diversos
projetos interessantes com sensores e atuadores, você começa a sentir
necessidade de entender realmente como essa placa funciona e como o
microcontrolador realiza a mágica de transformar o código de seu programa em
ações no mundo real.

Montar um ``Arduino Standalone'' (usar apenas o microcontrolador --- geralmente
o ATmega328P --- na protoboard sem a placa do Arduino) é um verdadeiro
\textbf{rito de passagem} no aprendizado de eletrônica, programação e sistemas
embarcados: é o momento em que você deixa de ser apenas um usuário do Arduino
para entender como a engenharia por trás dele funciona.

Pense no Arduino como uma bicicleta com rodinhas de treinamento: você consegue
fazer muitas coisas legais e interessantes mas chega um momento em que, para
avançar, as rodinhas de treinamento precisam ser removidas. Montar seu primeiro
Arduino Standalone equivale à seu primeiro passeio de bicicleta sem as rodinhas
de treinamento.

Existem diversas razões pelas quais isso é interessante e importante:

\begin{itemize}
  \item \textbf{Desmistificação do hardware}: quando usamos o Arduino, muitas
        coisas estão pré-configuradas e ocultas para nós. Ao montar o circuito
        independente na protoboard você aprenderá na prática a função de
        diversos componentes essenciais como o cristal oscilador, a regulação de
        tensão para a alimentação do microcontrolador e a montagem do circuito
        de reset. Além disso você entenderá de verdade o que o microcontrolador
        precisa obrigatoriamente para funcionar (alimentação estável, clock,
        reset em nível definido, capacitores de desacoplamento, etc.) e o que é
        opcional e fornecido pelo Arduino apenas como um ``conforto'' para o
        usuário (conversor USB-Serial, LEDs, etc.).
  \item \textbf{Eficiência energética}: a placa do Arduino possui componentes
        que constantemente consomem energia (LEDs, conversor USB-Serial) mesmo
        que seu código não esteja fazendo nada. Ao eliminar todos os componentes
        não essenciais e configurar corretamente o microcontrolador (usando o
        modo de \ingles{sleep}) você pode fazer seu Arduino Standalone funcionar
        por vários meses usando apenas duas pilhas AA (um Arduino completo
        acabaria com as pilhas em alguns dias).
  \item \textbf{Custo e permanência}: imagine que você criou um pequeno sistema
        de automação para o portão da garagem de sua casa. Você deixaria seu
        Arduino preso lá no portão para sempre? Correndo o risco de um fio se
        desconectar com o tempo? Você vai perder uma placa relativamente cara e
        que tem diversas outras possibilidades de usos além do portão?
        Provavelmente não. O melhor é que você utilize apenas o microcontrolador
        e os componentes básicos para fazer seu sistema de automação funcionar
        e, ao montar o Arduino Standalone, você aprenderá a como fazer isso.
  \item \textbf{Transição para o produto final}: nenhum produto final acabado
        tem um Arduino colado dentro dele. Produtos reais utilizam apenas o
        microcontrolador e os demais componentes eletrônicos soldados em uma
        placa de circuito impressa (PCB\footnote{\ingles{Printed Circuit
        Board} (PCB).}) fabricada profissionalmente. Aprender a montar um
        Arduino Standalone é um passo intermediário obrigatório para que, no
        futuro, você aprenda a projetar suas próprias PCB profissionais. Você
        entenderá na prática a diferença entre uma placa de desenvolvimento e a
        versão de produção final.
  \item \textbf{Flexibilidade}: ao usar o microcontrolador independente você não
        fica preso aos \qty{5.0}{\volt} e \qty{16}{\mega\hertz} do Arduino:
        você pode rodar o chip em \qty{3.3}{\volt} e se comunicar com sensores
        modernos sem conversores de nível, ou pode usar o clock interno de
        \qty{8}{\mega\hertz} do microcontrolador e liberar mais dois pinos
        digitais extras para uso.
  \item \textbf{Entendimento do datasheet}: para entender certas funções mais
        avançadas do microcontrolador você será forçado a ler o datasheet e
        procurar coisas como os tipos de clock possíveis, configuração de fuses
        e a organização dos pinos.
  \item \textbf{Liberdade de forma e conectores}: ao usar o Arduino você é
        obrigado a utilizar os conectores pin header fêmea que, apesar de bons
        para prototipagem, são ruins para o produto final (não oferecem fixação
        segura dos fios jumper). Com o microcontrolador independente você é quem
        decide qual o tamanho e o formato da placa final, e quais serão os
        conectores mais adequados para seu produto (bornes de parafuso,
        conectores Molex e outros).
  \item \textbf{Exercício de engenharia}: montar um Arduino Standalone lhe
        proporcionará um exercício de engenharia de sistemas embarcados
        completo pois você precisará montar e/ou entender a parte elétrica
        (fonte, reguladores, capacitores, proteções), a parte digital (clock,
        reset, sinais de entrada/saída, interface com sensores e atuadores), o
        firmware (fuses, bootloader, programação ISP\footnote{%
        \ingles{In-System Programming} (ISP), também chamado por
        \ingles{In-Circuit Serial Programming} (ICSP).}, debug), documentação
        (projeto esquemática, projeto da PCB) e fabricação do seu produto final.
\end{itemize}

Em resumo, ao montar seu próprio Arduino Standalone você estará dando seus
primeiros passos para se tornar um verdadeiro engenheiro de sistemas
embarcados. Obviamente esse é um objetivo ambicioso demais para ser tratado em
uma pequena série de artigos como esta, então resolvi me limitar aqui aos
aspectos básicos e mais simples de criar um Arduino Standalone na esperança de
que munido desta base você seja capaz de aprender cada vez mais sobre sistemas
embarcados e o fantástico mundo dos microcontroladores.

Neste primeiro artigo faremos a montagem básica do Arduino Standalone, ou seja,
a montagem inicial do microcontrolador na PCB, incluindo a fonte de alimentação,
o circuito de reset, e o circuito de clock externo. Meu objetivo é que este
artigo seja um tutorial completo e detalhado o suficiente para você montar seu
microcontrolador independente de forma segura\footnote{Este artigo é, na
verdade, o tutorial que eu queria ter tido acesso quando eu tentei montar um
Arduino Standalone pela primeira vez.}. Alguns pressupostos que estou assumindo:

\begin{itemize}
  \item Montaremos um Arduino Standalone usando o microcontrolador Microchip
        ATmega328P-PU, que é o utilizado no Arduino Uno.
  \item Utilizaremos como fonte de alimentação uma bateria ou fonte de
        \qty{9}{\volt}, conectada diretamente à protoboard.
  \item Nossa montagem será a mais fiel possível ao Arduino Uno\footnote{Este
        item parece um contra-senso pois acabamos de citar todas as vantagens de
        abandonar o Arduino e usar o microcontrolador independente. Sim, isso é
        meio que um contra-senso mesmo, mas estou partindo do pressuposto que
        você está montando seu primeiro Arduino Standalone e manter a
        arquitetura o mais próxima possível do Arduino, nesse momento,
        facilitará sua aprendizagem.}.
\end{itemize}
  
