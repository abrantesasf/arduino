%%%%%%%%%%%%%%%%%%%%%%%%%%%%%%%%%%%%%%%%%%%%%%%%%%%%%%%%%%%%%%%%%%%%%%%%%%%%%%%%
\section{Montagem}
\label{sec:montagem}


%%%%%%%%%%%%%%%%%%%%%%%%%%%%%%%%%%%%%%%%
\subsection{Alimentação}
\label{sec:montagem-alimentacao}

A primeira cosia que montaremos na protoboard é o circuito regulador de tensão,
que receberá os \qty{9}{\volt} da bateria e fornecerá \qty{5}{\volt} para
alimentar o microcontrolador e demais componentes. O esquema desse circuito
regulador está apresentado na Figura~\ref{fig:reg-tensao}:

\begin{figure}[H]
\centering
\caption{Circuito regulador de tensão}
\label{fig:reg-tensao}
\vspace{-0.2cm}
  \includegraphics[scale=0.28]{imagens/regulador_tensao.png}
\end{figure}

O diodo D1 (1N4007) é uma proteção contra polaridade reversa. Se você
inadvertidamente conectar a bateria ao contrário, o diodo bloqueará a corrente
impedindo que ela flua para o regulador L7805CV e para o resto do circuito
(potencialmente queimando os componentes). Esse diodo causa uma queda de tensão
de até \qty{1.1}{\volt} dependendo do modelo e do fabricante\footnote{Note que
devido à queda de tensão causada pelo diodo D1, a tensão indicada entre os
capacitores C1 e C2 ($+$\qty{9}{\volt}) não é totalmente correta, deveria ser
algo em torno de \qty{8}{\volt}. Em nome da simplicidade e para deixar a
explicação mais didática, preferi deixar o circuito com essa pequena
imprecisão.}.

Os capacitores eletrolíticos de \qty{47}{\micro\farad} atuam fazendo
filtragem de ruídos de baixa freqüência e como reservatório de energia
(nessa função costumam ser chamados de \ingles{bulk capacitors}). O capacitor C1
(na entrada) garante que a tensão na entrada do regulador L7805CV permaneça
estável mesmo se a bateria tiver uma queda momentânea de tensão. Já o capacitor
C4 (na saída) melhora a resposta a transientes de carga e evita que a tensão de
\qty{5}{\volt} caia bruscamente (por exemplo, se um módulo que de repente puxa
mais corrente).

Os capacitores cerâmicos de \qty{100}{\nano\farad} atuam como capacitores de
desacoplamen\-to e fazem filtragem de ruídos de alta freqüência. O capacitor C2
(na entrada) filtra ruídos que venham da bateria, diodo ou fios. Já o capacitor
C3 (na saída) filtra ruídos que venham do L7805CV ou da própria carga. Esses
capacitores devem ser localizados o mais próximo possível do L7805CV e atual
como um reservatório de energia local de resposta rápida.

O L7805CV é um regulador linear de tensão que aceita de \qty{7}{\volt} a
\qty{25}{\volt} na entrada, fornece \qty{5}{\volt} na saída, suporta correntes
de até \qty{1.5}{\ampere}, e conta com mecanismos de proteção contra sobrecarga
térmica (se ele esquentar demais ele ``desliga'') e contra curtos circuitos. Ele
é um regulador excelente para projetos de Arduino Standalone e pequenos projetos
de sistemas embarcados de iniciantes, mas tem uma desvantagem: ele precisa
dissipar bastante calor e pode esquentar muito dependendo da carga que será
consumida no circuito. O cálculo da potência (em Watt) que o regulador precisará
dissipar é dado por:
\begin{equation}
  \label{eq:potencia}
  P = (T_i - T_o) \times I\,,
\end{equation}
onde $P$ é potência em Watt, $T_i$ é a tensão na entrada, $T_o$ é a
tensão na saída e $I$ é a corrente consumida em seu projeto (tensões em Volt e
corrente em Ampere). Voltaremos a isso posteriormente.

Inicie a montagem parafusando o clip da bateria no conector borne de duas vias,
e coloque o borne na protoboard verificando corretamente se os polos positivos e
negativos do borne estão alinhados com as trilhas vermelha e azul da protoboard
(não coloque a bateria ainda). Depois coloque o diodo 1N4007 conectando a trilha
positiva a uma linha da protoboard (lembre-se de que o diodo é polarizado, você
precisa colocar o ânodo\footnote{Lembre-se de que o \textbf{ânodo} é o polo
positivo e o \textbf{cátodo} é o polo negativo. O ânodo do diodo 1N4007 é o lado
sem nenhuma marcação, e o cátodo é o lado que contém uma pequena faixa branca ou
cinza.} do diodo na trilha positiva de \qty{9}{\volt} e o cátodo em uma linha da
protoboard) conforme a Figura~\ref{fig:borne-diodo}:

\begin{figure}[H]
\centering
\caption{Borne e diodo}
\label{fig:borne-diodo}
\vspace{-0.2cm}
  \includegraphics[scale=0.7]{imagens/alimentacao3.jpeg}
\end{figure}

Depois, com um fio rígido, faça uma conexão entre a trilha negativa a uma linha
da protoboard ao lado do diodo, conforme a Figura~\ref{fig:diodo-gnd}:

\begin{figure}[H]
\centering
\caption{Conexão do \ingles{ground} (GND)}
\label{fig:diodo-gnd}
\vspace{-0.2cm}
  \includegraphics[scale=1.0]{imagens/alimentacao4.jpeg}
\end{figure}

O próximo passo é conectar os capacitores de entrada. Coloque o capacitor
eletrolítico de \qty{47}{\micro\farad} na protoboard, conectando o ânodo do
capacitor com o cátodo do diodo, e o cátodo do capacitor na mesma linha do GND
da protoboard, conforme a Figura~\ref{fig:capacitor47entrada}. Cuidado ao fazer
essa conexão pois capacitores eletrolíticos são polarizados e uma conexão
invertida pode até causar uma pequena explosão (o cátodo do capacitor é indicado
por uma faixa branca com sinais negativos).

\begin{figure}[H]
\centering
\caption{Capacitor \qty{47}{\micro\farad} de entrada}
\label{fig:capacitor47entrada}
\vspace{-0.2cm}
  \includegraphics[scale=1.0]{imagens/alimentacao5.jpeg}
\end{figure}

Agora devemos conectar o capacitor cerâmico de \qty{100}{\nano\farad} de
entrada. Esse capacitor não é polarizado e pode ser colocado de qualquer lado,
conforme a Figura~\ref{fig:capacitor100entrada}:

\begin{figure}[H]
\centering
\caption{Capacitor \qty{100}{\nano\farad} de entrada}
\label{fig:capacitor100entrada}
\vspace{-0.2cm}
  \includegraphics[scale=0.9]{imagens/alimentacao6.jpeg}
\end{figure}

O resultado até aqui, com a colocação dos capacitores de entrada, está mostrado
na Figura~\ref{fig:capent} (página~\pageref{fig:capent}).

\begin{figure}[H]
\centering
\caption{Capacitores de entrada}
\label{fig:capent}
\vspace{-0.2cm}
  \includegraphics[scale=0.6]{imagens/alimentacao7.jpeg}
\end{figure}

O próximo passo é colocar o regulador L7805CV. Conforme o \ingles{datasheet} do
meu regulador os pinos são dispostos como na Figura~\ref{fig:pinosl7805} (talvez
você tenha que conferir a pinagem do modelo exato do regulador que você está
utilizando; na dúvida consulte o \ingles{datasheet}).

\begin{figure}[H]
\centering
\caption{Pinos do L7805CV}
\label{fig:pinosl7805}
\vspace{-0.2cm}
  \includegraphics[scale=0.4]{imagens/alimentacao20.png}
\end{figure}

Para a colocação do L7805CV na protoboard: alinhe o pino de \ingles{input} do
regulador com a linha de \qty{9}{\volt}, e o pino de \ingles{ground} do
regulador com a linha GND, conforme a Figura~\ref{fig:reguladorproto}:

\begin{figure}[H]
\centering
\caption{Regulador L7805CV}
\label{fig:reguladorproto}
\vspace{-0.2cm}
  \includegraphics[scale=0.9]{imagens/alimentacao8.jpeg}
\end{figure}

Para termos mais espaço para trabalhar, vamos conectar o GND e a saída de
\qty{5}{\volt} do regulador no outro lado da protoboard, conforme a
Figura~\ref{fig:extendergnd5v}:

\begin{figure}[H]
\centering
\caption{GND e \qty{5}{\volt}}
\label{fig:extendergnd5v}
\vspace{-0.2cm}
  \includegraphics[scale=1.0]{imagens/alimentacao9.jpeg}
\end{figure}

Coloque agora mais um capacitor cerâmico de \qty{100}{\nano\farad} entre o GND e
o \qty{5}{\volt}, conforme a Figura~\ref{fig:capacitor100saida}. Idealmente esse
capacitor deveria ficar o mais próximo possível do regulador de tensão mas pela
falta de espaço na protoboard vamos deixá-lo do outro lado mesmo.

\begin{figure}[H]
\centering
\caption{Capacitor de \qty{100}{\nano\farad} da saída}
\label{fig:capacitor100saida}
\vspace{-0.2cm}
  \includegraphics[scale=1.1]{imagens/alimentacao10.jpeg}
\end{figure}

O segundo capacitor eletrolítico de \qty{47}{\micro\farad} deve ser colocado
agora. Alinhe o ânodo do capacitor com a linha de \qty{5}{\volt} e o cátodo do
capacitor com a linha de GND, conforme a Figura~\ref{fig:capacitor47saida}:

\begin{figure}[H]
\centering
\caption{Capacitor \qty{47}{\micro\farad} de saída}
\label{fig:capacitor47saida}
\vspace{-0.2cm}
  \includegraphics[scale=0.9]{imagens/alimentacao11.jpeg}
\end{figure}

Agora use fios rígidos para conectar a saída de \qty{5}{\volt} e o GND do
regulador de tensão às trilhas vermelha e azul da protoboard, conforme a
Figura~\ref{fig:alimentacao5v} e a Figura~\ref{fig:alimentacao5vb}:

\begin{figure}[H]
\centering
\caption{Alimentação de \qty{5}{\volt} da protoboard}
\label{fig:alimentacao5v}
\vspace{-0.2cm}
  \includegraphics[scale=0.9]{imagens/alimentacao12.jpeg}
\end{figure}

\begin{figure}[H]
\centering
\caption{Alimentação de \qty{5}{\volt} da protoboard}
\label{fig:alimentacao5vb}
\vspace{-0.2cm}
  \includegraphics[scale=1.0]{imagens/alimentacao13.jpeg}
\end{figure}

Coloque a bateria e verifique se as conexões estão corretas com um multímetro:
ao medir a tensão na trilha inferior da protoboard, você deve obter
\qty{5}{\volt} conforme a Figura~\ref{fig:verificacao5v} (a tensão medida, de
\qty{4.99}{\volt} está dentro da margem de tolerância do regulador).

\begin{figure}[H]
\centering
\caption{Verificação da tensão}
\label{fig:verificacao5v}
\vspace{-0.2cm}
  \includegraphics[scale=1.0]{imagens/alimentacao15.jpeg}
\end{figure}

Agora já temos o circuito regulador de tensão montado e funcionando
perfeitamente. Um pequeno inconveniente desse circuito é que não há nenhuma
indicação visual de que o circuito está energizado. Assim, como um refinamento
opcional, vamos colocar um LED indicativo de que o circuito está
energizado e ligado. Coloque o ânodo do LED diretamente na trilha de
\qty{9}{\volt} da protoboard e o cátodo do LED em uma linha vazia conforme a
Figura~\ref{fig:ledind}:

\begin{figure}[H]
\centering
\caption{LED indicativo}
\label{fig:ledind}
\vspace{-0.2cm}
  \includegraphics[scale=1.0]{imagens/alimentacao16.jpeg}
\end{figure}

Agora vamos colocar o resistor de \qty{680}{\ohm} entre o cátodo do LED e a
trilha de GND da protoboard, conforme a Figura~\ref{fig:resled}:

\begin{figure}[H]
\centering
\caption{Resistor do LED}
\label{fig:resled}
\vspace{-0.2cm}
  \includegraphics[scale=1.0]{imagens/alimentacao17.jpeg}
\end{figure}

Esse resistor serve para limitar a corrente que percorrerá o LED que, segundo as
especificações do \ingles{datasheet}, suporta correntes de até
\qty{20}{\milli\ampere} com uma \ingles{voltage forward} de \qty{2.1}{\volt}. A
corrente que percorre o LED será então de:
\begin{equation}
  \label{eq:corrente-led}
  I = \cfrac{V}{R} = \cfrac{\qty{9.0} -\ \qty{2.1}{}}{680} =
  \cfrac{\qty{6.9}{}}{680} \approx \qty{10}{\milli\ampere}
\end{equation}

A corrente de \qty{10}{\milli\ampere} é segura para o LED mas representa um
consumo de bateria a mais para o projeto (uma bateria alcalina comum tem
capacidade entre \qtyrange{400}{600}{\milli\ampere\hour} e o LED iria esgotar a
bateria entre \qtyrange{40}{60}{\hour}). Se você não precisa ou não quer esse
consumo extra, você pode remover o LED completamente. Particularmente eu gosto
de uma indicação visual de que o circuito está energizado. Uma alternativa para
diminuir o consumo da bateria é colocar resistores maiores para diminuir a
corrente que percorre esse LED.

O circuito de alimentação está quase terminado. Coloque a bateria e verifique se
o LED está funcionando conforme a Figura~\ref{fig:regled}:

\begin{figure}[H]
\centering
\caption{Regulador com LED indicativo}
\label{fig:regled}
\vspace{-0.2cm}
  \includegraphics[scale=0.8]{imagens/alimentacao18.jpeg}
\end{figure}

Por fim, como estou usando uma protoboard com quatro trilhas de alimentação
independente, conectei as trilhas onde quero disponibilizar \qty{5}{\volt} e
coloquei pequenas etiquetas para identificar as tensões. A protoboard com o
circuito de alimentação finalizado está mostrada na
Figura~\ref{fig:alimentacaofim}.

\begin{figure}[H]
\centering
\caption{Circuito de alimentação finalizado}
\label{fig:alimentacaofim}
\vspace{-0.2cm}
  \includegraphics[scale=0.8]{imagens/alimentacao19.jpeg}
\end{figure}

Ainda é possível fazer algumas outras pequenas melhorias no circuito de
alimentação como, por exemplo, colocar um diodo adicional entre os pinos de
\ingles{input} e \ingles{output} do regulador (para proteger contra tensão
reversa) ou acrescentar outros diodos em série com o diodo D1 para diminuir a
tensão que o regulador recebe. Entretanto essas melhorias são opcionais para
esse nosso pequeno projeto e, por isso, não serão feitas.


%%%%%%%%%%%%%%%%%%%%%%%%%%%%%%%%%%%%%%%%
\subsection{Microcontrolador e reset}
\label{sec:montagem-microcont}

O próximo passo é fazer a montagem do microcontrolador ATmega328P-PU e do
circuito de reset (que faz o microcontrolador reiniciar). Uma das primeiras
coisas que você deve ter em mente é que os números dos pinos do Arduino são
completamente diferentes dos números reais dos pinos do microcontrolador (o
Arduino abstrai muito da complexidade do hardware e esconde as configurações e
nomes verdadeiros dos componentes do microcontrolador). Em qualquer projeto de
Arduino Standalone você precisa ter sempre à mão um mapeamento entre os pinos do
microcontrolador e os nomes ``bonitinhos'' que o Arduino utiliza. A
Figura~\ref{fig:pinagem} traz esse mapeamento:

\begin{figure}[H]
\centering
\caption{Mapeamento ATmega328P $\times$ Arduino}
\label{fig:pinagem}
\vspace{-0.2cm}
  \makebox[\textwidth][c]{%
    \includegraphics[scale=0.28]{imagens/pinagem_atmega_arduino.png}%
  }
\end{figure}

Uma explicação mais detalhada sobre a pinagem do ATmega328P é necessária. O
microcontrolador tem $28$ pinos físicos, $14$ de cada lado, numerados
fisicamente de \qtyrange{1}{28}{}. O pino $1$ é identificado por estar do lado
esquerdo de um entalhe em forma de meia-lua em uma das extremidades do
microcontrolador.

Esse Microcontrolador tem três portas de I/O\footnote{I/O:
\ingles{input}/\ingles{output}; entrada/saída.} de propósito geral que são
chamadas de \textbf{Port B}, \textbf{Port C} e \textbf{Port D}. Cada uma dessas
portas pode ter até o máximo de oito pinos de I/O. Esses pinos são genericamente
chamados de \ingles{General-Purpose I/O} (GPIO). Por exemplo: a Port C tem $7$
pinos, chamados de PC0 até PC6, que estão conectados aos pinos físicos $1$ e
$23$--$28$ do microcontrolador (pinos de cor amarela na
Figura~\ref{fig:pinagem}). Note também que cada um desses pinos pode exercer
mais de uma função no microcontrolador, por exemplo: o pino PC5 pode exercer
três funções diferentes, ADC5, SCL, PCINT13 (identificadas na pinagem do
microcontrolador, logo ao lado do nome do pino). Essa nomenclatura do
microcontrolador é mascarada pelos nomes utilizados no Arduino, de forma que o
pino PC5 do microcontrolador é conhecido por pino A5 no Arduino; o pino PD5 do
microcontrolador é conhecido por pino D5 no Arduino; o pino PB4 do
microcontrolador é conhecido por pino D12 no Arduino; e assim por diante. A
Figura~\ref{fig:pinagem} identifica todo o mapeamento dos pinos reais do
ATmega328 com os nomes utilizados pelo Arduino. Note também que mesmo na
nomenclatura do Arduino os pinos podem exercer mais de uma função, por exemplo:
o pino A5 do Arduino também pode ser utilizado como pino D19 ou como pino
SCL\footnote{Um dos pinos utilizados em comunicação \ingles{Inter-Integrated
Circuit} (I$^2$C)}.

O estudo detalhado de cada um desses pinos e das funções que eles podem exercer
está fora do escopo deste artigo mas é necessário que você tenha a
Figura~\ref{fig:pinagem} por perto ao montar circuitos de Arduino Standalone
para que você consiga substituir as conexões feitas diretamente no Arduino pelas
conexões a serem feitas diretamente no microcontrolador.

Para começar a montagem do microcontrolador, coloque-o na protoboard,
identifique o pino $1$ e faça a ligação do VCC (que fornece \qty{5}{\volt} ao
ATmega328P) no pino $7$, e do GND no pino $8$, conforme demonstrado na
Figura~\ref{fig:microcont-vccgnd}. Tenha \textbf{muito cuidado} para não errar
as conexões aos pinos do microcontrolador!

\begin{figure}[H]
\centering
\caption{VCC e GND}
\label{fig:microcont-vccgnd}
\vspace{-0.2cm}
  \includegraphics[scale=0.8]{imagens/microcontrolador1.jpeg}
\end{figure}

Agora coloque um capacitor cerâmico de \qty{100}{\nano\farad} entre o pino $7$
(VCC) e o GND, conforme a Figura~\ref{fig:microcont-capvcc}. Isso é um
\textbf{capacitor de desacoplamento} e é muito importante para a estabilidade do
circuito. Ele realiza duas funções importantes para que o ATmega328P não trave
ou reinicialize sozinho: a) reservatório local de energia para evitar quedas de
tensão quando a comutação dos transístores do microcontrolador exigir picos
rápidos de corrente; e b) filtragem de ruídos de alta freqüência gerados por
motores, relés ou outros componentes, pois o capacitor oferece um caminho fácil
para o ruído ir direto para GND, desacoplando o ruído da alimentação principal
do microcontrolador.

\begin{figure}[H]
\centering
\caption{Capacitor entre VCC e GND}
\label{fig:microcont-capvcc}
\vspace{-0.2cm}
  \includegraphics[scale=0.8]{imagens/microcontrolador2.jpeg}
\end{figure}

O próximo passo é colocar um indutor axial de \qty{10}{\micro\henry} entre a
linha de \qty{5}{\volt} e o pino AVCC (pino $20$), e um capacitor cerâmico de
\qty{100}{\nano\farad} entre o pino AVCC (pino $20$) e o GND, conforme a
Figura~\ref{fig:microcont-avcc}:

\begin{figure}[H]
\centering
\caption{Conexão do AVCC}
\label{fig:microcont-avcc}
\vspace{-0.2cm}
  \includegraphics[scale=0.38]{imagens/microcontrolador4b.png}
\end{figure}

Por que esse conjunto de indutor mais capacitor é necessário? O pino AVCC é a
fonte de tensão para o conversor analógico-digital (ADC) do ATmega328 (que lê
tensões analógicas nos pinos PC0 até PC5), e esse conversor precisa de uma
tensão de referência perfeitamente estável para fazer as leituras e comparar os
sinais. O conjunto indutor mais capacitor forma um \textbf{Filtro LC} (Filtro
Indutor-Capacitor) cuja função é proteger a exatidão das leituras analógicas. O
indutor atua como um filtro passa-baixa e age como uma entrada seletiva: para
corrente contínua o indutor é praticamente um fio comum, e deixa passar os
\qty{5}{\volt} para alimentar o ADC; para ruídos de alta freqüência o indutor
apresenta alta impedância (resistência) e ``bloqueia'' as oscilações rápidas e
ruídos gerados por chaveamento digital ou outras fontes. O capacitor ajuda o
indutor e ``engole'' qualquer resíduo que o indutor deixar passar. O resultado é
que a tensão no pino AVCC é uma energia ``limpa'' e estável.

Agora faça a conexão de um outro capacitor de \qty{100}{\nano\farad} entre o
pino AREF (pino 21) e o GND, conforme exibido na
Figura~\ref{fig:microcont-caparef}:

\begin{figure}[H]
\centering
\caption{Capacitor para AREF}
\label{fig:microcont-caparef}
\vspace{-0.2cm}
  \includegraphics[scale=0.8]{imagens/microcontrolador5.jpeg}
\end{figure}

Esse capacitor no AREF merece uma explicação um pouco mais detalhada. O pino
AREF é a referência de tensão analógica para as leituras do conversor
analógico-digital do microcontrolador. Essa tensão de referência é como uma
``régua'' a partir da qual as tensões analógicas são medidas: o ADC funciona
comparando a tensão de entrada em um pino analógico com a tensão de referência
em AREF. Essa tensão de referência pode ser configurada para:

\begin{itemize}
  \item \textbf{AREF Externo}: nessa configuração o microcontrolador exige que
        você conecte ao pino AREF uma fonte de tensão de comparação externa. Se
        você não conectar uma tensão de referência externa as leituras
        analógicas serão incorretas. Você usaria essa configuração se, por
        exemplo, você está utilizando um sensor que trabalha em
        \qty{3.3}{\volt}: nesse caso você deve conectar uma tensão de
        \qty{3.3}{\volt} em AREF para que essa tensão seja a referência de
        comparação do ADC com as leituras feitas a partir do
        sensor. \textbf{Atenção:} o uso de AREF externo implica em certos
        cuidados no código, em especial você não pode fazer nenhuma leitura
        analógica sem antes informar e configurar o microcontrolador para
        utilizar corretamente essa tensão de referência externa. Note que mesmo
        que você utilize uma referência externa de tensão o capacitor de
        \qty{100}{\nano\farad} ainda é necessário entre AREF e GND.
  \item \textbf{AREF Interno em \qty{1.1}{\volt}}: nessa configuração você não
        precisa conectar nenhuma tensão de referência no pino AREF, o
        microcontrolador utilizará uma tensão interna de \qty{1.1}{\volt} como
        referência analógica. Essa é uma situação não muito comum pois a maioria
        dos sensores analógicos utilizados trabalha com tensões de
        \qty{5}{\volt} ou \qty{3.3}{\volt}. O capacitor entre AREF e GND
        continua sendo necessário.
  \item \textbf{AREF em AVCC}: nessa configuração estamos dizendo ao
        microcontrolador para usar como tensão de referência analógica a tensão
        de entrada no pino AVCC (que conectamos aos \qty{5}{\volt} filtrados e
        limpos com o Filtro LC). O capacitor entre AREF e GND continua sendo
        necessário.
\end{itemize}

Por padrão o ATmega328P vem configurado de fábrica para usar ``AREF Externo'',
ou seja, nós somos obrigados a conectar uma tensão de referência no pino
AREF. Mas por que não fizemos isso e só conectamos um capacitor cerâmico de
\qty{100}{\nano\farad} nesse pino? Esse é um outro exemplo de como o Arduino
esconde as complexidades do hardware e do microcontrolador para simplificar um
pouco as coisas para usuários iniciantes: ao compilar um código pela IDE do
Arduino, usando as funções e bibliotecas do próprio Arduino, o Arduino
automaticamente altera as configurações do ATmega328P para usar ``AREF em AVCC''
e ajusta outras coisas para que o ADC utilize como tensão de referência os
\qty{5}{\volt} filtrados e limpos conectados ao pino AVCC\footnote{Se você esta
programando o microcontrolador em C puro (AVR C), sem usar as bibliotecas do
Arduino, você é obrigado a ajustar manualmente em seu código a configuração do
AREF desejada e a fazer alguns outros ajustes para que o conversor
analógico-digital funcione corretamente. Isso é feito alterando-se a
configuração de alguns registradores do microcontrolador, mas isso foge ao
escopo deste artigo.}. O capacitor conectado entre AREF e GND ajuda a
estabilizar ainda mais a tensão de referência para o conversor
analógico-digital.

Ao usar ``AREF em AVCC'' não devemos conectar mais nenhuma outra fonte de tensão
no pino AREF ($21$), apenas o capacitor de filtragem, pois a tensão de
comparação será fornecida apenas por AVCC.

O próximo passo na montagem do microcontrolador é conectar o pino GND ($22$) à
linha de GND da protoboard, conforme a Figura~\ref{fig:microcont-gndadc}. O pino
de GND ($22$) é conectado internamento ao pino de GND ($8$), mas ele fica
disponível para o ADC e, às vezes, é encontrado com a identificação AGND.

\begin{figure}[H]
\centering
\caption{GND para ADC}
\label{fig:microcont-gndadc}
\vspace{-0.2cm}
  \includegraphics[scale=0.7]{imagens/microcontrolador6.jpeg}
\end{figure}

O pino RESET ($1$) do microcontrolador deve ser conectado, através de um
resistor de \qty{10}{\kilo\ohm}, à trilha de \qty{5}{\volt} da protoboard,
conforme a Figura~\ref{fig:microcont-pinoreset}. Esse pino funciona da seguinte
maneira: quando ele está em nível lógico ``HIGH'' (\qty{5}{\volt}) o
microcontrolador está funcionando normalmente; mas se ele for colocado em nível
lógico ``LOW'' (\qty{0}{\volt}) o microcontrolador é reiniciado.

\begin{figure}[H]
\centering
\caption{Pino de RESET}
\label{fig:microcont-pinoreset}
\vspace{-0.2cm}
  \includegraphics[scale=0.7]{imagens/microcontrolador7.jpeg}
\end{figure}

Nesse momento já temos todas as conexões básicas de alimentação do
microcontrolador prontas. O próximo passo agora é montar um circuito de reset
para o microcontrolador pois para podermos gravar um novo programa o
microcontrolador deve ser reiniciado. Vamos montar o circuito de RESET.

Coloque um diodo 1N4148 conectado entre o pino RESET ($1$) e a linha de
\qty{5}{\volt} da protoboard da seguinte maneira: o cátodo do diodo deve ficar
conectado aos \qty{5}{\volt} e o ânodo do diodo deve ficar conectado ao pino de
RESET ($1$), ou seja: o diodo está ``invertido'' e não deixa passar corrente da
alimentação para o pino de RESET (a corrente obrigatoriamente passa pelo
resistor). Veja a Figura~\ref{fig:microcont-resetdiodo} para entender como esse
diodo deve ser conectado.

\begin{figure}[H]
\centering
\caption{Circuito de reset: diodo}
\label{fig:microcont-resetdiodo}
\vspace{-0.2cm}
  \includegraphics[scale=0.7]{imagens/microcontrolador8.jpeg}
\end{figure}

Qual o propósito desse diodo? Bem, nesse momento ele não tem função praticamente
nenhuma, ele está tecnicamente ``inútil''\footnote{Exceto, talvez, por uma
pequena proteção contra descarga eletrostática se alguém tocar no botão com o
dedo carregado de eletricidade estática: nesse caso o diodo ajuda a desviar
essa energia para a fonte ao invés de danificar a porta de RESET do
microcontrolador.} e poderia até ser removido do circuito sem grandes
prejuízos. Mesmo assim é uma boa prática deixar esse diodo já conectado. O diodo
1N4148 é conhecido como \textbf{diodo de sinal} ou diodo de chaveamento. Isso
significa que ele é extremamente veloz ao passar do estado ``bloqueando'' para o
estado ``conduzindo'' energia. Ele está aí por um motivo de proteção futura se
resolvermos usar um adaptador FTDI/USB-Serial para a programação do
microcontrolador.

Para programar o microcontrolador, em muitos projetos de Arduino Standalone, nós
usamos um adaptador FTDI/USB-Serial. Para que a IDE do Arduino consiga enviar o
código automaticamente, ela precisa reiniciar o microcontrolador. Isso é feito
ligando-se o adaptador ao pino RESET através de um capacitor de
\qty{100}{\nano\farad} (que não está na protoboard, mas costuma ser adicionado
depois se realmente formos usar o adaptador FTDI/USB-Serial).

Mas um problema que ocorre ao usar esse adaptador é que ele força a tensão no
pino de RESET ir para \qty{0}{\volt}, reiniciando o microcontrolador, mas,
quando a tensão volta para os \qty{5}{\volt} da linha de tensão da protoboard, a
tensão armazenada no capacitor se soma à tensão da linha de alimentação. Isso
pode gerar um pico de tensão momentâneo no pino RESET que pode chegar até
\qty{10}{\volt}, causando efeitos inesperados ou prejudiciais no
ATmega328P. O diodo 1N4148 age como uma válvula de escape (chamado de
\textbf{clamping diode}) e funciona do seguinte modo: se a tensão no pino RESET
subir acima de aproximadamente \qty{5.7}{\volt}, ele começa a conduzir e desvia
esse excesso de tensão para a linha de \qty{5}{\volt} da protoboard, mantendo o
pino se RESET seguro.

Mesmo que o diodo 1N4148 não exerça função praticamente nenhuma em nosso projeto
no momento, se adicionarmos um adaptador FTDI/USB-Serial para a programação do
microcontrolador no futuro o circuito de RESET já estará protegido e, por isso,
é melhor deixar esse diodo já conectado.

Para finalizarmos o circuito de reset temos que colocar um botão na protoboard
que, ao ser pressionado, coloca o pino de RESET ($1$) em ``LOW'' e faz o
microcontrolador se reiniciar. Para isso coloque um botão ao lado do
microcontrolador, e ligue o botão no pino de RESET ($1$) e ao GND, conforme a
Figura~\ref{fig:microcont-resetbotao}. Esse é o circuito de reset.

\begin{figure}[H]
\centering
\caption{Circuito de reset: botão}
\label{fig:microcont-resetbotao}
\vspace{-0.2cm}
  \includegraphics[scale=0.7]{imagens/microcontrolador9.jpeg}
\end{figure}

Nesse momento nosso microcontrolador já está totalmente montado e pronto: se
quiséssemos já poderíamos usar o Arduino Standalone sem problema
algum. Entretanto, nessa montagem, somos forçados a utilizar o clock interno do
microcontrolador, que não é tão veloz quando comparado a um clock
externo. Vamos portanto acrescentar um circuito de clock externo ao nosso
Arduino Standalone.


%%%%%%%%%%%%%%%%%%%%%%%%%%%%%%%%%%%%%%%%
\subsection{Clock externo}
\label{sec:montagem-clock}

Acrescentar um clock externo ao nosso Arduino Standalone não é difícil. Em
primeiro lugar vamos colocar um resistor de \qty{1}{\mega\ohm} entre os pinos
$9$ e $10$ do microcontrolador (pinos XTAL1 e XTAL2), conforme a
Figura~\ref{fig:microcont-clockresistor}.

\begin{figure}[H]
\centering
\caption{Clock externo: resistor}
\label{fig:microcont-clockresistor}
\vspace{-0.2cm}
  \includegraphics[scale=0.6]{imagens/clock1.jpeg}
\end{figure}

De modo simplificado esse resistor ajuda o cristal oscilador a começar a
funcionar. A conexão é difícil de fazer na protoboard pois o resistor não cabe
entre os pinos $9$ e $10$: a solução é dobrar os pinos de contato e deixar o
resistor na diagonal.

Coloque dois capacitores cerâmicos de \qty{22}{\pico\farad} para conectar os
pinos $9$ e $10$ ao GND da protoboard, conforme a
Figura~\ref{fig:microcont-clockcapacitores}.

\begin{figure}[H]
\centering
\caption{Clock externo: capacitores}
\label{fig:microcont-clockcapacitores}
\vspace{-0.2cm}
  \includegraphics[scale=0.7]{imagens/clock3.jpeg}
\end{figure}

Também de modo simplificado a função desses capacitores é filtrar ruídos
indesejados e garantir que a oscilação do cristal gere uma onda senoidal limpa e
na freqüência correta.

Por fim, coloque um cristal oscilador de \qty{16}{\mega\hertz} entre os pinos
$9$ e $10$ (XTAL1 e XTAL2), conforme a
Figura~\ref{fig:microcont-clockcristal}. A colocação do cristal também é difícil
pois ele não cabe adequadamente na protoboard e precisa ser colocado de lado.

\begin{figure}[H]
\centering
\caption{Clock externo: cristal}
\label{fig:microcont-clockcristal}
\vspace{-0.2cm}
  \includegraphics[scale=0.7]{imagens/clock4.jpeg}
\end{figure}

Agora sim, de fato, terminamos a montagem do microcontrolador, com o circuito de
reset e o clock externo. Você deve ter algo semelhante à Figura

\begin{figure}[H]
\centering
\caption{Microcontrolador, reset e clock}
\label{fig:microcont-reset-clock}
\vspace{-0.2cm}
  \includegraphics[scale=0.7]{imagens/clock5.jpeg}
\end{figure}


%%%%%%%%%%%%%%%%%%%%%%%%%%%%%%%%%%%%%%%%
\subsection{Montagem completa}
\label{sec:montagem-completa}

Se você seguiu à risca as instruções de montagem, seu Arduino Standalone deve
se parecer com a Figura~\ref{fig:ardstand-completo}, abaixo:

\begin{figure}[H]
\centering
\caption{Arduino Standalone completo}
\label{fig:ardstand-completo}
\vspace{-0.2cm}
  \includegraphics[scale=0.9]{imagens/standalone_pronto.jpeg}
\end{figure}

Para referência, as figuras abaixo trazem todos os diagramas esquemáticos de
nosso Arduino Standalone.

\begin{figure}[H]
\centering
\caption{Esquemático: regulador de tensão}
\label{fig:esq-reg-ten}
\vspace{-0.2cm}
  \includegraphics[scale=0.28]{imagens/esquematico_regulador_tensao.png}
\end{figure}

\begin{figure}[H]
\centering
\caption{Esquemático: microcontrolador}
\label{fig:esq-micro}
\vspace{-0.2cm}
  \includegraphics[scale=0.6]{imagens/esquematico_atmega.png}
\end{figure}

\begin{figure}[H]
\centering
\caption{Esquemático: circuito de reset}
\label{fig:esq-reset}
\vspace{-0.2cm}
  \includegraphics[scale=0.3]{imagens/esquematico_reset.png}
\end{figure}

\begin{figure}[H]
\centering
\caption{Esquemático: clock externo}
\label{fig:esq-clock}
\vspace{-0.2cm}
  \includegraphics[scale=0.3]{imagens/esquematico_clock.png}
\end{figure}
