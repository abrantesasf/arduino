%%%%%%%%%%%%%%%%%%%%%%%%%%%%%%%%%%%%%%%%%%%%%%%%%%%%%%%%%%%%%%%%%%%%%%%%%%%%%%%%
\section{Próximos passos}
\label{sec:proximos}

O que podemos fazer com nosso Arduino Standalone? No momento praticamente
nada\ldots\ antes de podermos utilizar o microcontrolador na protoboard para
nossos projetos de sistemas embarcados, precisamos aprender a programá-lo
diretamente na protoboard.

Se você está com pressa em ver seu Arduino Standalone funcionando,
pode utilizar o seguinte artifício: utilize o Arduino normalmente para o seu
projeto, fazendo o protótipo conforme você já está acostumado. Ao terminar,
\textbf{remova cuidadosamente} o microcontrolador do Arduino e coloque na
protoboard fazendo exatamente as conexões que mostramos aqui. Depois reproduza
na protoboard todas as conexões que você tinha quando estava usando a placa do
Arduino (lembre-se de que os pinos do Arduino não correspondem exatamente aos
pinos do microcontrolador --- utilize a Figura~\ref{fig:pinagem} para se
orientar). Agora você tem um microcontrolador já programado para seu
projeto. Claro, essa não é a melhor solução: ficar removendo e recolocando o
pode acabar danificando o microcontrolador.

As melhores soluções para a programação de seu Arduino Standalone são usar um
adaptador FTDI/USB-Serial ou um programador AVR dedicado. Falarei sobre isso em
artigos futuros.
