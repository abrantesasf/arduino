%%%%%%%%%%%%%%%%%%%%%%%%%%%%%%%%%%%%%%%%%%%%%%%%%%%%%%%%%%%%%%%%%%%%%%%%%%%%%%%%
% Arduino Standalone I
% Por: Abrantes Araújo Silva Filho
%      abrantesasf@computacaoraiz.com.br
%%%%%%%%%%%%%%%%%%%%%%%%%%%%%%%%%%%%%%%%%%%%%%%%%%%%%%%%%%%%%%%%%%%%%%%%%%%%%%%%


%%%%%%%%%%%%%%%%%%%%%%%%%%%%%%%%%%%%%%%%%%%%%%%%%%%%%%%%%%%%%%%%%%%%%%%%%%%%%%%%
%%% Classe do documento
\documentclass[12pt]{article}

%%%%%%%%%%%%%%%%%%%%%%%%%%%%%%%%%%%%%%%%%%%%%%%%%%%%%%%%%%%%%%%%%%%%%%%%%%%%%%%%
%%% Preâmbulo com todas as outras outras chamadas para todos os outros packages
%%% e o que mais for necessário
\input{utils/preambulo.tex}

%%%%%%%%%%%%%%%%%%%%%%%%%%%%%%%%%%%%%%%%%%%%%%%%%%%%%%%%%%%%%%%%%%%%%%%%%%%%%%%%
%%% Ajuste do layout, espaçamento de linhas e etc.:
\geometry{a4paper, portrait}
%\onehalfspacing

%%%%%%%%%%%%%%%%%%%%%%%%%%%%%%%%%%%%%%%%%%%%%%%%%%%%%%%%%%%%%%%%%%%%%%%%%%%%%%%%
%%% Configurações para as propriedades do PDF:
\hypersetup{
   %hidelinks,           % Comente para web, descomente para impressão
   colorlinks=true,      % True para web, False para impressão
   pdftitle={Arduino Standalone: I},
   pdfauthor={Abrantes Araújo Silva Filho},
   pdfsubject={Montagem de Arduino em protoboard},
   pdfkeywords={arduino, atmega328, protoboard},
   pdfinfo={
      CreationDate={}, % Ex.: D:AAAAMMDDHH24MISS
      ModDate={}       % Ex.: D:AAAAMMDDHH24MISS
   }
}
\input{utils/pdfconfig}

%%%%%%%%%%%%%%%%%%%%%%%%%%%%%%%%%%%%%%%%%%%%%%%%%%%%%%%%%%%%%%%%%%%%%%%%%%%%%%%%
%%% Compilação condicional de seções
%\includeonly{}

%%%%%%%%%%%%%%%%%%%%%%%%%%%%%%%%%%%%%%%%%%%%%%%%%%%%%%%%%%%%%%%%%%%%%%%%%%%%%%%%
%%% Começa o documento
\begin{document}

%%%%%%%%%%%%%%%%%%%%%%%%%%%%%%%%%%%%%%%%%%%%%%%%%%%%%%%%%%%%%%%%%%%%%%%%%%%%%%%%
%%% Front matter

% Bookmark para a capa do artigo
\pdfbookmark[1]{Início}{titulo}

% Título (se necessário um subtítulo, usar quebra de linha e o tamanho large):
\title{\textbf{Arduino Standalone: I}}
%\bookmark[dest=titulo,level=chapter]{Title Page}

% Autor e data:
\author{Abrantes Araújo Silva Filho}
\date{2025-11-18}

% Gera os títulos:
\maketitle

% Abstract:
%\renewcommand{\abstractname}{\textbf{LEIA COM ATENÇÃO!}}
\abstract{\noindent Este é o primeiro de uma série de artigos que discute e
mostra, em detalhes, como montar um Arduino Standalone, ou seja, um Arduino
montado apenas em uma protoboard. Neste artigo aprenderemos a fazer a montagem
da fonte de alimentação, do microcontrolador e de outros componentes
fundamentais, como o cristal oscilador. Em outros artigos desta série
discutiremos as opções disponíveis para a programação do microcontrolador
diretamente na protoboard.}

% Sumário:
\pdfbookmark[1]{Sumário}{sumario}
\tableofcontents

%%%%%%%%%%%%%%%%%%%%%%%%%%%%%%%%%%%%%%%%%%%%%%%%%%%%%%%%%%%%%%%%%%%%%%%%%%%%%%%%
%%% Main matter
\newpage
%%%%%%%%%%%%%%%%%%%%%%%%%%%%%%%%%%%%%%%%%%%%%%%%%%%%%%%%%%%%%%%%%%%%%%%%%%%%%%%%
\section{Introdução}
\label{sec:intro}

Depois de algum tempo utilizando o Arduino e ganhando experiência em diversos
projetos interessantes com sensores e atuadores, você começa a sentir
necessidade de entender realmente como essa placa funciona e como o
microcontrolador realiza a mágica de transformar o código de seu programa em
ações no mundo real.

Montar um ``Arduino Standalone'' (usar apenas o microcontrolador --- geralmente
o ATmega328P --- na protoboard sem a placa do Arduino) é um verdadeiro
\textbf{rito de passagem} no aprendizado de eletrônica, programação e sistemas
embarcados: é o momento em que você deixa de ser apenas um usuário do Arduino
para entender como a engenharia por trás dele funciona.

Pense no Arduino como uma bicicleta com rodinhas de treinamento: você consegue
fazer muitas coisas legais e interessantes mas chega um momento em que, para
avançar, as rodinhas de treinamento precisam ser removidas. Montar seu primeiro
Arduino Standalone equivale à seu primeiro passeio de bicicleta sem as rodinhas
de treinamento.

Existem diversas razões pelas quais isso é interessante e importante:

\begin{itemize}
  \item \textbf{Desmistificação do hardware}: quando usamos o Arduino, muitas
        coisas estão pré-configuradas e ocultas para nós. Ao montar o circuito
        independente na protoboard você aprenderá na prática a função de
        diversos componentes essenciais como o cristal oscilador, a regulação de
        tensão para a alimentação do microcontrolador e a montagem do circuito
        de reset.
  \item \textbf{Eficiência energética}: a placa do Arduino possui componentes
        que constantemente consomem energia (LEDs, conversor USB-Serial) mesmo
        que seu código não esteja fazendo nada. Ao eliminar todos os componentes
        não essenciais e configurar corretamente o microcontrolador (usando o
        modo de \ingles{sleep}) você pode fazer seu Arduino Standalone funcionar
        por vários meses usando apenas duas pilhas AA (um Arduino completo
        acabaria com as pilhas em alguns dias).
  \item \textbf{Custo e permanência}: imagine que você criou um pequeno sistema
        de automação para o portão da garagem de sua casa. Você deixaria seu
        Arduino preso lá no portão para sempre? Correndo o risco de um fio se
        desconectar com o tempo? Você vai perder uma placa relativamente cara e
        que tem diversas outras possibilidades de usos além do portão?
        Provavelmente não. O melhor é que você utilize apenas o microcontrolador
        e os componentes básicos para fazer seu sistema de automação funcionar
        e, ao montar o Arduino Standalone, você aprenderá a como fazer isso.
  \item \textbf{Transição para o produto final}: nenhum produto final acabado
        tem um Arduino colado dentro dele. Produtos reais utilizam apenas o
        microcontrolador e os demais componentes eletrônicos soldados em uma
        placa de circuito impressa (PCB\footnote{Do inglês: ``\ingles{Printed
        Circuit Board}''.}) fabricada profissionalmente. Aprender a montar um
        Arduino Standalone é um passo intermediário obrigatório para que, no
        futuro, você aprenda a projetar suas próprias PCB profissionais.
  \item \textbf{Flexibilidade}: ao usar o microcontrolador independente você não
        fica preso aos \qty{5.0}{\volt} e \qty{16}{\mega\hertz} do Arduino:
        você pode rodar o chip em \qty{3.3}{\volt} e se comunicar com sensores
        modernos sem conversores de nível, ou pode usar o clock interno de
        \qty{8}{\mega\hertz} do microcontrolador e liberar mais dois pinos
        digitais extras para uso.
\end{itemize}

%%%%%%%%%%%%%%%%%%%%%%%%%%%%%%%%%%%%%%%%%%%%%%%%%%%%%%%%%%%%%%%%%%%%%%%%%%%%%%%%
\section{Materiais}
\label{sec:materiais}

Nosso Arduino Standalone terá, na protoboard, os seguintes grandes componentes:

\begin{enumerate}
  \item \textbf{Alimentação}: usaremos uma bateria ou fonte de alimentação de
        \qty{9}{\volt} mas o ATmega328P suporta, no máximo
        \qty{5.5}{\volt}. Precisaremos então de um circuito regulador de tensão
        que receba os \qty{9}{\volt} e entregue \qty{5}{\volt} para o
        microcontrolador.
  \item \textbf{Microcontrolador}: é o ATmega328P em si. Você pode remover o
        microprocessador do Arduino, ou comprar um novo.
  \item \textbf{Clock externo}: usaremos um circuito de clock externo de
        \qty{16}{\mega\hertz} com um cristal oscilador.
  \item \textbf{Circuito de reset}: um botão na protoboard permitirá que o
        microcontrolador seja reiniciado rapidamente, sem a necessidade de
        desconectar a fonte de energia.
\end{enumerate}

A organização desses grandes componentes na protoboard será aproximadamente a
exibida na figura~\ref{fig:orgnizacao_componentes}, a seguir:

\begin{figure}[H]
\centering
\caption{Organização geral dos componentes}
\label{fig:orgnizacao_componentes}
\vspace{-0.2cm}
%\fbox{
  \includegraphics[scale=0.18]{imagens/protoboard3.jpg}
%}\
%\footnotesize{Fonte: xxx}
\end{figure}

Os materiais e componentes necessários são os seguintes:

\begin{itemize}
  \item \textbf{Alimentação}: a lista a seguir e a
        figura~\ref{fig:comp-alimentacao} mostram os componentes que
        utilizaremos para a alimentação do microcontrolador. Alguns itens são
        opcionais mas recomendados neste estágio inicial de seu aprendizado.
        \begin{itemize}[noitemsep]
          \item 1 regulador linear de tensão L7805CV
                % www.newark.com/stmicroelectronics/l7805cv/ldo-fixed-5v-1-5a-0-to-125deg/dp/26M0575
          \item 1 diodo 1N4007
                % www.newark.com/multicomp-pro/1n4007/diode-standard-recovery-1a-1kv/dp/65W8781
          \item 2 capacitores eletrolíticos de \qty{47}{\micro\farad}
                % newark.com/chemi-con/esmg500ell470mf11d/aluminum-electrolytic-capacitor/dp/23K5056
          \item 2 capacitores cerâmicos de \qty{100}{\nano\farad}
                % ???
          \item 1 LED vermelho de \qty{5}{\mm} (opcional, apenas para indicar
                que a protoboard está energizada)
                % www.newark.com/cree-led/c503b-rcn-cx0y0aa1/led-round-5mm-red-5-1cd-624nm/dp/08R2995
          \item 1 resistor de \qty{680}{\ohm} (opcional, apenas para indicar que
                a protoboard está energizada)
                % www.newark.com/multicomp-pro/mf50-680r/metal-film-resistor-680-ohm-500mw/dp/38K5245
          \item 1 conector borne de duas vias
                % www.newark.com/lumberg/kre-03/wire-to-board-terminal-block-3/dp/25M9958
          \item 1 clip para bateria de \qty{9}{\volt} e 1 bateria de \qty{9}{\volt}
                % www.newark.com/keystone/233/battery-strap-9v-wire-lead/dp/22C4351
                % www.newark.com/energizer/522bp-2/alkaline-zn-mno2-battery-9v/dp/17C5580
          \item 1 fonte de alimentação de \qty{9}{\volt} (\qty{1}{\ampere})
                (opcional: se você não quiser utilizar a bateria pode utilizar
                uma fonte como alimentação)
                % www.newark.com/pro-elec/28-19355/9vdc-1a-regulated-ac-power-adapter/dp/62X7077
          \begin{figure}[H]
          \centering
          \caption{Componentes para a alimentação}
          \label{fig:comp-alimentacao}
          \vspace{-0.2cm}
            \includegraphics[scale=0.7]{imagens/alimentacao.jpeg}
          %\footnotesize{Fonte: xxx}
          \end{figure}
        \end{itemize}
\end{itemize}


%%%%%%%%%%%%%%%%%%%%%%%%%%%%%%%%%%%%%%%%%%%%%%%%%%%%%%%%%%%%%%%%%%%%%%%%%%%%%%%%
%%% Apêndices
%\appendix
%\input{apend/placeholder}

%%%%%%%%%%%%%%%%%%%%%%%%%%%%%%%%%%%%%%%%%%%%%%%%%%%%%%%%%%%%%%%%%%%%%%%%%%%%%%%%
%%% Back matter
%\bibliography{utils/biblioteca}
%\printindex

%%%%%%%%%%%%%%%%%%%%%%%%%%%%%%%%%%%%%%%%%%%%%%%%%%%%%%%%%%%%%%%%%%%%%%%%%%%%%%%%
%%% Termina o documento
\end{document}

